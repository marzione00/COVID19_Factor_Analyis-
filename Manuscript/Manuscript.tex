%%%%%%%%%%%%%%%%%%%%%%%%%%%%%%%%%%%%%%%%%
% Arsclassica Article
% LaTeX Template
% Version 1.1 (1/8/17)
%
% This template has been downloaded from:
% http://www.LaTeXTemplates.com
%
% Original author:
% Lorenzo Pantieri (http://www.lorenzopantieri.net) with extensive modifications by:
% Vel (vel@latextemplates.com)
%
% License:
% CC BY-NC-SA 3.0 (http://creativecommons.org/licenses/by-nc-sa/3.0/)
%
%%%%%%%%%%%%%%%%%%%%%%%%%%%%%%%%%%%%%%%%%

%----------------------------------------------------------------------------------------
%	PACKAGES AND OTHER DOCUMENT CONFIGURATIONS
%----------------------------------------------------------------------------------------

\documentclass[
12pt, % Main document font size
a4paper, % Paper type, use 'letterpaper' for US Letter paper
oneside, % One page layout (no page indentation)
%twoside, % Two page layout (page indentation for binding and different headers)
headinclude,footinclude, % Extra spacing for the header and footer
BCOR5mm, % Binding correction
]{scrartcl}
\usepackage[english]{babel}
\usepackage{url}
\usepackage{graphicx}
\usepackage{subcaption}
\usepackage{float}
\usepackage{epigraph}
\usepackage{mathcomp}
\usepackage{textcomp}
\usepackage{amsmath}
\usepackage{url}
\usepackage{sectsty}
\usepackage[dvipsnames,table,xcdraw,svgnames]{xcolor}
\usepackage{listings}
\lstset{language=R,
    basicstyle=\small\ttfamily,
    stringstyle=\color{DarkGreen},
    otherkeywords={0,1,2,3,4,5,6,7,8,9},
    morekeywords={TRUE,FALSE},
    deletekeywords={data,frame,length,as,character},
    keywordstyle=\color{blue},
    commentstyle=\color{DarkGreen},
}

\usepackage[pdftex,
            pdfauthor={Marzio De Corato},
            pdftitle={A statistical analysis on factors that contributed/slowed down the spread of COVID-19},
            pdfsubject={Project for the exam: Machine learning, statistical learning, deep learning and artificial intelligenc},
            pdfproducer={Latex},
            pdfcreator={pdflatex}]{hyperref}

\input{structure.tex} % Include the structure.tex file which specified the document structure and layout
\sloppy
\hyphenation{Fortran hy-phen-ation} % Specify custom hyphenation points in words with dashes where you would like hyphenation to occur, or alternatively, don't put any dashes in a word to stop hyphenation altogether

%----------------------------------------------------------------------------------------
%	TITLE AND AUTHOR(S)
%----------------------------------------------------------------------------------------

\begin{document}


\title{\normalfont{A statistical analysis on factors that contributed/slowed down the spread of COVID-19}} % The article title

\subtitle{Project for the exam: Machine learning, statistical learning, deep learning and artificial intelligence} % Uncomment to display a subtitle


\author{Marzio De Corato} % The article author(s) - author affiliations need to be specified in the AUTHOR AFFILIATIONS block

\date{\today} % An optional date to appear under the author(s)

%----------------------------------------------------------------------------------------





%----------------------------------------------------------------------------------------
%	HEADERS
%----------------------------------------------------------------------------------------

\renewcommand{\sectionmark}[1]{\markright{\spacedlowsmallcaps{#1}}} % The header for all pages (oneside) or for even pages (twoside)
%\renewcommand{\subsectionmark}[1]{\markright{\thesubsection~#1}} % Uncomment when using the twoside option - this modifies the header on odd pages
\lehead{\mbox{\llap{\small\thepage\kern1em\color{halfgray} \vline}\color{halfgray}\hspace{0.5em}\rightmark\hfil}} % The header style

\pagestyle{scrheadings} % Enable the headers specified in this block

%----------------------------------------------------------------------------------------
%	TABLE OF CONTENTS & LISTS OF FIGURES AND TABLES
%----------------------------------------------------------------------------------------

\maketitle % Print the title/author/date block

\newpage
\setlength\epigraphwidth{.7\textwidth}
\epigraph{ "Haec ratio quondam morborum et mortifer aestus \\
finibus in Cecropis funestos reddidit agros \\
vastavitque vias, exhausit civibus urbem. \\
nam penitus veniens Aegypti finibus ortus, \\
aeëra permensus multum camposque natantis, \\
incubuit tandem populo Pandionis omni." \\
Lucretius, De Rerum Natura VI \\
\vspace{5mm}
Twas such a manner of disease, 'twas such \\
Mortal miasma in Cecropian lands \\
Whilom reduced the plains to dead men's bones,\\
Unpeopled the highways, drained of citizens \\
The Athenian town. For coming from afar,\\
Rising in lands of Aegypt, traversing \\
Reaches of air and floating fields of foam, \\
At last on all Pandion's folk it swooped \\
(As translated by William Ellery Leonard)
}



\newpage


%----------------------------------------------------------------------------------------
%	ABSTRACT
%----------------------------------------------------------------------------------------

\section*{Abstract} % This section will not appear in the table of contents due to the star (\section*)
The anisotropic spread of the virus \textit{SARS-CoV-2} raised a scientific challenge about the features that prevent or instead increase its diffusion. In order to get a bird's-eye view on this issue concerning to the first wave (up to 24/08/2020) the principal component analysis (PCA) was performed on the following clusters: the Italian provinces, the Italian regions, and a selected set of 150 countries. Such inspection allowed to investigate the disease diffusion not only for a clusters sets of different sizes, but also with different features since, in principle the indices that are available for a set of provinces are different with respect to the ones that are available for a set of countries. It was found that while some features such as the mean income, the GDP and the public transport are positively correlated with a higher number of COVID19 cases, due to the fact that they increase the connectivity of the agents, others that limit the latter, as the unemployment slow the diffusion. Moreover we found that, concerning the health service, the most important factor that is negatively correlated with the COVID19 cases is the normalized number of general practitioner and their normalized number of visits, while a less important role is played by the health quality indices.

\nocite{*}
\setcounter{tocdepth}{2} % Set the depth of the table of contents to show sections and subsections only

\tableofcontents % Print the table of contents

%\listoffigures % Print the list of figures

%\listoftables % Print the list of tables


\newpage % Start the article content on the second page, remove this if you have a longer abstract that goes onto the second page


\section{Introduction} \label{introduction}
The recent pandemic diffusion of the virus \textit{SARS-CoV-2} connected to the \textit{COronaVIrus Disease 19} raised up a scientific issue about the different spread ratio among the different population clusters such as cities,regions and states. This behaviour can be ascribed to the fact that the features of these clusters, such as the densities, the public transports as well as the individual mean incomes are very heterogeneous \cite{sebhatu2020explaining,skorka2020macroecology}. Furthermore the individual policies undertaken by the decision makers of these clusters have a large effect on the spread of the virus \cite{block2020social}. A good tool for an analysis about the correlations of the spread of the virus with the cluster feature is the Principal Component Analysis (PCA): this approach allows to visualize in to a single 2D plot the main statistical correlation of a data set. It is worth nothing that this type of analysis represent only the first step for the identification of causal correlation between the features: the findings obtained via PCA should be validated with a model; however this second step is out from the aim of this paper \footnote{But may be considered in to a future one} . In this work three different type of cluster were considered: the Italian provinces (\textit{province}), the Italian regions (\textit{regioni}), and 150 countries. The choice of the of Italian provinces was motivated by the fact that this the smallest cluster for which the daily cases of COVID19 are easily available, on the other hand the regions provide aggregate data (in particular about the sanitary system) that are not easily retrievable for the provinces. Finally the inclusion of the states is aimed to see if the statistical correlation obtained for the smaller cluster were still valid in a macroscopic scale \footnote{For this last cluster the lock-down policy and its effect was not considered, because the data collection of the policies for each 150 countries was too time consuming, however this analysis may be an outlook for a future update of this work}. 

\section{Data description} \label{Data_description}
In this section a brief overview about the data analysed is provided: in particular, for each cluster,a brief description is provided  about the data gathering and assembling. 

\subsection{Temporal interval}

In this work, for Italy, only the first wave of contagion was considered: this choice was motivated by the fact that the two waves of contagion (the first one starting from February to the end of August, and the second one starting from the end of September) may have a different spread distributions due to the different locations of initial cases. Moreover the second wave, up to the date of this work, is still active, therefore a complete analysis on it should be postponed to its end \footnote{This does not exclude, in principle, that the statistical correlations obtained in this work can be the same for the second wave}. On the other side for the other countries, a better approach should have considered the procedure described before: however for many of them the end of the first wave and the staring of the second one it is not so clear as in Italy; on this bases the at state level no distinction of waves was considered \footnote{But will be considered in future development}.   


\subsection{Provinces} 
For this cluster the cumulative number of cases of COVID19 up to 24/08/2020 was obtained from the Github website of \textit{Protezione Civile} \cite{github-protezionecivile} . The overall number of total cases was than divided (normalized) by the population number of its province. The population number (2019) was retrieved from Istituto Nazionale di Statistica (ISTAT) \cite{ISTAT}. The size of the province was also considered in order to evaluate the density. Since the diffusion of a virus is causally correlated to connectivity and public mobility services \cite{lloyd2001viruses,kraemer2020effect} we considered also a set of economic indices in order to evaluate the industrialization and the wealth of each province: these are the mean income for person (2019), the public transport (2012 measured as demand for resident), the private transport (2012 measured as cars for resident), the pollution and the unemployment ( 2019). The first one was taken from \textit{Ministero dell'Economia e delle Finanze} as reported in the following website \cite{MEF} while the other ones were obtained from the ISTAT databases \cite{ISTAT}.

\subsection{Regions}
As for the provinces also for this cluster the cumulative number of cases of COVID19 was taken from from \cite{github-protezionecivile} up to 24/08/2020. In this case the same source provides the overall number of tests. The unemployment rate as well the mean income were also available from ISTAT \cite{ISTAT} (2019). Since the Italian constitution delegates partially the management of health services to the regions, new features connected to the latter are available. Here the following ones were considered: the number of resident citizens for general practitioner (\textit{medico di base}, the normalized number of structures for hospitalization, the normalized number of medical guards (\textit{guardie mediche}) multiplied by $10^{5}$ and their normalized number of visit performed multiplied by $10^{5}$. These data were retrieved from the statistical yearbook of national health service 2017 (\textit{Annuario Statistico del sistema sanitario nazionale 2017} published on the Italian minister of health \cite{AnnuarioSSN2017}. Finally also  the marks for the essential assistance levels \textit{livelli essenziali di assistenza} were included as provided by the Italian minister of health for 2017 \cite{LeaRank} . These represent an evaluation of the health services for each region according to the Italian Government. A full description of the feature analysed can be found here \cite{LeaDesc}

\subsection{Countries}

For this type of cluster the overall number of cases and deaths was taken from from the World Health Organization \cite{whoCases} up to 27/08/2020. These number were normalized with the data about the total population for state provided by World Bank \cite{worldBank} up to 2019. The GDP (up to 2019) and Universal Health Coverage Index \cite{whoUHC} (up to 2017) for each country was also taken from the latter source \cite{worldBank} and than normalized. Finally the National Health Expenditure (NHA) (normalized) up to 2017, the Traffic Morality (normalized and scaled with a factor $10^{5}$) up to 2016, the Pollution Mortality (normalized), the PM 2.5 concentration ($\mu g/m^{3}$) up to 2016 were taken from the WHO databases \cite{whoDb}. It is worth nothing that selection for the counties considered in this work were the ones for which all these indicators were available. 


\section{Theoretical Background} \label{Theoretical Background}
As pointed out by Mackay \cite{mackay2003information} the basic idea of unsupervised learning is to mime human behaviour is to find regularities in data and group them. Among the different techniques of unsupervised learning \cite{james2013introduction} here we considered the PCA: in this section the basic ideas of this tool will be focused following the approach proposed by James et al. \cite{james2013introduction}. One of the main aim of the PCA is to plot $n$ observations with $p$ features in only one 2D plot,with the least possible loss of information, instead of $\binom{p}{2}$ 2D plots. To do so the PCA chooses, among the possible axis formed by the linear normalized combination of the features, the two associated with the largest variance, given that these two axis are orthogonal. Thus, supposing that the means of the features are null, the problem is to find the coefficients (usually called loadings) $\phi_{ij}$ that solves the following optimization problem  \cite{james2013introduction}:
\begin{equation}
max\left\lbrace \dfrac{1}{n} \sum_{i=1}^{n} \left(\sum^{p} _{j=1} \phi_{j1}x_{ij} \right)^{2}   \right\rbrace
\end{equation}
with the constraint  \cite{james2013introduction}
\begin{equation}
\sum_{j=1}^{p}\phi_{j1}^{2}=1
\end{equation}
The loadings define the principal components (that have to be orthogonal)  \cite{james2013introduction}: 
\begin{equation}
z_{i1}=\phi_{11}x_{i1}+\phi_{21}x_{i2}+...+\phi_{p1}x_{ip}
\end{equation}
If the all the principal components are taken in to account no information is lost, however the complexity of it due to high dimensionality is the same of the starting one; however if we sacrifice the p-2 component, that have a lower variance with respect to the first two, the complexity is dramatically reduced to a 2D plot. In this case the toll paid is the lost of the information contained in the p-2 components. From a computational point of view this task can be performed with the standard techniques used for solving a eigenvalue problem. Once the two principal component are found, the data can plotted into the new coordinate set; the advantage of this new representation lies on the fact that if the first two principal component loading vectors are plotted a bird's-eye view of the statistical correlation between the features is obtained: basically the cosine of the angle between the loadings approximates the statistical correlation, while the position of the data with respect to the loadings will plot its feature. In this way an intuitive representation of the data, their features and the correlation among them is provided. Finally this can be accompanied by a plot of the correlation matrix as it was done in this work. 

\section{Results and discussion}

For each cluster the following procedure was followed: first the correlation matrix was calculated together with its heath-map, then the PCA was performed on the scaled data. For this procedure an histogram about the importance of the components, a 2D plot of the loading vector normalized on unitary circle, and the full 2D PCA plot with the data were reported. 

\subsection{Provinces} \label{sub_prov}

From the inspection of the plots \ref{Province_corr_matrix},\ref{Province_FULL_Variables-PCA} and \ref{Provinces_PCA_FULL} it is possible to point out that the public transport, the density, the normalized cumulative cases and the mean income are positively correlated. This statistical correlations are causally validated by different previous publications \cite{neiderud2015urbanization,world2010hidden,gangemi2020rich,weyers2008low}. The argument that explains this behaviour is that a higher density, a higher public transport demand and a higher income increase the rate of contact between the individuals: the first one since people are closer and so a contact between has a higher probability with respect to a low density areas (such as a rural context), the second one since when individuals use the public transport are very close each other (indeed there is no correlation between the cases and private transport), the third one because rich people can spend more money for social events or perhaps in to travels. This fact is corroborated by the negative statistical correlation between cases and unemployment (in this case also the fact that a worker has a higher mobility due to the fact that he/she has to reach the place of work must be considered). One can be at first confused by the fact that the air quality have a strong correlation with the private transport and the public transport has a different direction: this can ascribed to the fact that in rural provinces, were the air quality is higher and the public transport services are reduced,  people are forced to own a private vehicle, while in urban provinces they can consider to use only the public transport. Finally note that looking to the plot \ref{Province_FULL_Variances} the PCA analysis reported in plots \ref{Province_FULL_Variables-PCA} and \ref{Provinces_PCA_FULL} captures almost the 40 $\%$ of variance. 


\begin{figure}[h]
\begin{center}
\includegraphics[scale=1]{Pic/Province_FULL_CorrMatrix.pdf}
\caption{The Correlation matrix for the Italian provinces accompanied by its heat-map as generated by the following dataset: local unemployment (2019), local private transport (2012 number of cars for 1000 residents), the air quality (2012) ,the local public transport (2012 measured as the demand for resident), the density  (2019 measured as resident for Km$^{2}$, the cumulative cases up to 26/08/2020 and the mean income (2019 measured in kEUR). Note that the public transport, the density the cumulative cases and the mean income are highly correlated. On the other hand the unemployment is negatively correlated to this first cluster}
\label{Province_corr_matrix}
\end{center}
\end{figure}

\begin{figure}[h]
\begin{center}
\includegraphics[scale=1]{Pic/Province_FULL_Variances.pdf}
\caption{Percentage of variance for each component of the dataset concerning the provinces. Note that using the first two, almost the 60 $\% $ of variance is captured, as consequence the information lost is almost the 40 $\%$ }
\label{Province_FULL_Variances}
\end{center}
\end{figure}

\begin{figure}[h]
\begin{center}
\includegraphics[scale=1]{Pic/Province_FULL_Variables-PCA.pdf}
\caption{2D plot of the loading vector for the dataset of provinces normalized on the unitary circle. The features, the date of their collection and unit of measure are the same of Fig. \ref{Province_FULL_Variables-PCA}}
\label{Province_FULL_Variables-PCA}
\end{center}
\end{figure}

\begin{figure}[h]
\begin{center}
\includegraphics[scale=1]{Pic/Provinces_PCA_FULL.pdf}
\caption{The dataset of provinces plotted on the two principal components that capture the 60 $\% $ of variance. The provinces are labelled with their vehicle registration abbreviation. The features, the date of their collection and unit of measure are the same of Fig. \ref{Province_corr_matrix} }
\label{Provinces_PCA_FULL}
\end{center}
\end{figure}

\clearpage

\subsection{Regions}

Looking to the Fig. \ref{Regions_CorrMatrix},\ref{Regioni_PCA_loadings} and \ref{Regions_FULL_PCA} we found a positive correlation between the Mean Income and the Cases, as for the provinces,. This relation has the same explanation that was found in the previous paragraph. Furthermore here, since the overall number of tests are available as well as the number of deaths for COVID19, we see that these two normalized quantities are positively correlated with the cases. The causal connection for these three quantities is simple: in order to validate that a patient is infected a test is required, and a part of the infected patient are going to die for COVID19. The correlation matrix also highlights that there is a positive correlation between the number of people for general practitioner and the number of cases; moreover a higher number of normalized visit is negatively correlated with the number of cases and the mortality. The causal reason for this statistical correlation is currently under discussion but an interesting perspective, at a journalistic level, was performed on a Italian newspaper \cite{24plus}: the idea of the author is that the general practitioners play the role of sensor/sentinel for the COVID19 cases; as long as their number and their presence in all the cities is reduced, the health service is less sensitive to find new cases and the diffusion of the virus is favoured.  On the other side the quality index LEA,the normalized number of medical and the number of Public Structure Normalized  guards seems to play a weaker role. Finally it is worth nothing that, as shown in Fig. \ref{Region_variances} in the 2D plots \ref{Regioni_PCA_loadings} and \ref{Regions_FULL_PCA} only the 26.7 $ \% $ of information is lost.



\begin{figure}[h]
\begin{center}
\includegraphics[scale=1]{Pic/Regions_CorrMatrix.pdf}
\caption{The Correlation matrix for the Italian regions accompanied by its heat-map as generated by the following dataset: the normalized number of medical guards (2017), the normalized number of structures for hospitalization (2017), the normalized number of visit for medical guards (2017), the unemployment ratio (2019), the essential assistance levels (LEA - 2017, the number of normalized tests (24/08/2020), the mean income (2019), the population for general practitioner  (2017), the normalized deaths for COVID19 (24/08/2020), and the normalized number of cases (24/08/2020) }
\label{Regions_CorrMatrix}
\end{center}
\end{figure}


\begin{figure}[h]
\begin{center}
\includegraphics[scale=1]{Pic/Region_variances.pdf}
\caption{Percentage of variance for each component of the dataset concerning the provinces. Note that using the first two, the 73.3 $\% $ of variance is captured, as consequence the information lost is almost the 26.7 $\%$}
\label{Region_variances}
\end{center}
\end{figure}


\begin{figure}[h]
\begin{center}
\includegraphics[scale=1]{Pic/Regioni_PCA_loadings.pdf}
\caption{2D plot of the loading vector for the dataset of regions normalized on the unitary circle. The features, the date of their collection and unit of measure are the same of Fig. \ref{Regions_CorrMatrix}}
\label{Regioni_PCA_loadings}
\end{center}
\end{figure}


\begin{figure}[h]
\begin{center}
\includegraphics[scale=1]{Pic/Regions_FULL_PCA.pdf}
\caption{The dataset of regions plotted on the two principal components that capture the 73.3 $\% $ of variance. The features, the date of their collection and unit of measure are the same of Fig. \ref{Province_corr_matrix}}
\label{Regions_FULL_PCA}
\end{center}
\end{figure}


\clearpage

\subsection{Countries}

For this dataset it is possible to see from the inspection of Fig. \ref{World_CorrMatrix}, \ref{World_PCA_loadings} and \ref{World_FULL_PCA} that among the different features considered for the countries only the GDP pro-capita seems to be statistically correlated with the cases and the deaths. This fact can be explained with the same argument discussed in the section \ref{sub_prov}. In principle a possible objection to this link can be that only the richest countries can invest money for the tests for the validation of a COVID19 case: however it can be noted from plots that the normalized health expenditure as well the UHC index are basically orthogonal with respect to the normalized COVID19 cases and deaths. Moreover it is rummy, on the basis of recent papers , to find that the COVID cases and deaths are orthogonal to the pollution: this fact may be ascribed to the fact that the a smaller cluster should be considered; however it is worth nothing that in the literature the a strong causal relation is far from being validated \cite{contini2020does}. Finally, from the inspection of Fig. \ref{World_variances}, it is possibile to point out that the in the 2D plots \ref{Regioni_PCA_loadings} and \ref{Regions_FULL_PCA} the percentage of variance reported is almost the 58 $ \% $ while the 42 $ \% $ is lost.



\begin{figure}[h]
\begin{center}
\includegraphics[scale=1]{Pic/CorrMatrix_WORLD.pdf}
\caption{The Correlation matrix for the a selected set of countries accompanied by its heat-map as generated by the following dataset: The PM 2.5 concentration ($\mu g/m^{3}$ 2016), the normalized traffic morality (2016), the normalized pollution mortality, the GDP pro-capita(2019), the normalized deaths for COVID19 (27/08/2020), the overall number of cases (27/08/2020), the density (2019), the normalised health expenditure (2017), and the Universal Health Coverage Index (2017)}
\label{World_CorrMatrix}
\end{center}
\end{figure}


\begin{figure}[h]
\begin{center}
\includegraphics[scale=1]{Pic/Variances-PCA_WORLD.pdf}
\caption{Percentage of variance for each component of the dataset concerning the selected countries. Note that using the first two, the 57.9 $\% $ of variance is captured, as consequence the information lost is 42.1 $\%$}
\label{World_variances}
\end{center}
\end{figure}


\begin{figure}[h]
\begin{center}
\includegraphics[scale=1]{Pic/PCA-Loadings_WORLD.pdf}
\caption{2D plot of the loading vector for the dataset of the selected countries normalized on the unitary circle. The features, the date of their collection and unit of measure are the same of Fig. \ref{World_CorrMatrix}}
\label{World_PCA_loadings}
\end{center}
\end{figure}


\begin{figure}[h]
\begin{center}
\includegraphics[scale=1]{Pic/World_FULLPCA.pdf}
\caption{The dataset of countries plotted on the two principal components that capture the 57.9 $\% $ of variance. The features, the date of their collection and unit of measure are the same of Fig. \ref{World_CorrMatrix}}
\label{World_FULL_PCA}
\end{center}
\end{figure}


\clearpage


\section{Conclusions}

The analysis of the features of the different clusters pinpointed the following features: 

\begin{itemize}
\item The normalized number of cases and deaths for COVID19 are positively correlated for provinces and regions with the mean income, while for the country the same argument hold for the GDP pro-capita. As validated by the previous literature (see the subsection \ref{sub_prov}) , this relation is causal due to the fact a higher income increase the connectivity of the agents
\item The unemployment is negatively correlated with the number of cases and deaths: this effect is entangled with the mean income; thus for a deeper analysis on the causal relationship, an econometric research that disentangle the mean income and the employment is required. 
\item The spread of COVID19 does not seems to be negatively correlated with the quality and quantity of hospitals (LEA, UHC, Number of public structure), in reverse the number of general practitioner and their visit is negatively correlated with the number of cases and deaths. This can be due to the fact the a large number of normalized general practitioner are able to detect and isolate the infected agents, and this slows the spread of a disease. However a deeper analysis on this statical correlation is required in order support this argument. 
\item The public transport is positively correlated with the normalized new cases: the causal reason for this is that the agents during the travel increase their interactions with other agents thus the probability to be infected is higher
\end{itemize}

\section{R code}
\definecolor{light-gray}{gray}{0.95}
\lstset{ columns=fullflexible, basicstyle=\ttfamily, backgroundcolor=\color{light-gray},xleftmargin=0.5cm,frame=lr,framesep=8pt,framerule=0pt,frame=single,breaklines=true, postbreak=\mbox{\textcolor{red}{$\hookrightarrow$}\space}}
\lstinputlisting[language=R]{R_code_unsup.R}
%------------------------------------------------



%----------------------------------------------------------------------------------------
%	BIBLIOGRAPHY
%----------------------------------------------------------------------------------------

\renewcommand{\refname}{\spacedlowsmallcaps{References}} % For modifying the bibliography heading

\bibliographystyle{unsrt}

\bibliography{bibliography.bib} % The file containing the bibliography

%----------------------------------------------------------------------------------------

\end{document}